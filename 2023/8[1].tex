

\begin{enumerate}
	\item In \figref{fig.circle1},the quadrilateral $PQRS$ circumscribes a circle.Here $PA+CS$ is equal to : (see \tabref{tab:1}) 

\begin{figure}[H]
	        \centering
		\input{./figs/fig01.tex}
		\caption{}
		\label{fig.circle1}
\end{figure}



 \begin{table}[!ht]
	        \centering
	         \setlength{\tabcolsep}{80pt}
 \renewcommand{\arraystretch}{2}
 
  \begin{tabular}{l c}
	  a)$QR$ & b)$PR$ \\
	  c)$PS$ & d)$PQ$
  \end{tabular}

		\caption{}
		\label{tab:1}
 \end{table}




\item In \figref{fig.circle2},$\vec{O}$ is the center of the circle.$AB$ and $AC$ are tangents drawn to the circle from point $\vec{ A}$.If $\angle BAC=65\degree$, then find the measure of $\angle BOC$.


	\begin{center}
\begin{figure}[H]
	        \centering
	        \input{./figs/fig0002.tex}
		\caption{}
		\label{fig.circle2}
        \end{figure}
	\end{center}




\item In \figref{fig.circle3},$\vec{ O}$ is the centre of the circle and $QPR$ is a tangent to it at $\vec{ P}$. Prove that $\angle QAP+ \angle APR$= 90\degree.
\begin{figure}[H]
	        \centering
	        \input{./figs/fig02.tex}
		\caption{}
		\label{fig.circle3}

        \end{figure}





\item In \figref{fig.circle4},$PQ$ is tangent to the circle centred at $\vec{ O}$.If $\angle AOB= 95\degree$, then the measure of $\angle ABQ$ will be (see \tabref{tab:tab:2})
\begin{figure}[H]
	        \centering
	        \input{./figs/fig03.tex}
		\caption{}
		\label{fig.circle4}
        \end{figure}


\begin{table}[!ht]
	        \centering
	        \setlength{\tabcolsep}{80pt}
 \renewcommand{\arraystretch}{2}
 
  \begin{tabular}{l c}
       A)47.5\degree & B)42.5\degree\\
       C)85\degree& D)95\degree
  \end{tabular}

		\caption{}
		\label{tab:tab:2}
        \end{table}






\item
  \begin{enumerate}
	  \item Two tangents $TP$ and $TQ$ are drawn between to a circle with centre $\vec{O}$ from an external point $\vec{T}$ (\figref{fig.circle5}). Prove that $\angle PTQ = 2 \angle OPQ$.
\begin{figure}[h]
	        \centering
	        \input{./figs/fig5.tex}
		\caption{}
		\label{fig.circle5}
        \end{figure}





\item In \figref{fig.circle6},a circle is inscribed in a quadrilateral $ABCD$ in which $\angle B =90\degree$.If $AD=17cm,AB=20cm$ and $DS=3cm$, then find the radius of the circle.

\begin{figure}[h]
	        \centering
	        \input{./figs/fig6.tex}
		\caption{}
		\label{fig.circle6}
\end{figure}
   \end{enumerate}






\item The discus throw is an event in which an athlete attempts to throw a discus (as shown in \figref{fig.circle0}).The athlete spins anti-clockwise around one and a half times through a circle, then releases the throw. When released, the discus travels along tangent to the circular spin orbit.


\begin{figure}[H]	
	        \centering
		\includegraphics[width=\columnwidth]{./figs/fig0.png}
		\caption{}
		\label{fig.circle0}
\end{figure}



In \figref{fig.circle7}, $AB$ is one such tangent to a circle of radius 75 cm.Point $\vec{ O}$ is centre of the circle and $\angle ABO= 30\degree$.$PQ$ is parallel to $OA$.



\begin{figure}[H]
	        \centering
	        \input{./figs/fig7.tex}
		\caption{}
		\label{fig.circle7}
        \end{figure}



Based on above information:

           \begin{enumerate}
		   \item find the length of $AB$.
		   \item find the length of $OB$.
		   \item find the length of $AP$.
         \end{enumerate}


find the length of $PQ$.



\item In \figref{fig.circle8},$TA$ is a tangent to the circle with centre $\vec{O}$ such that $OT=4cm$, $\angle OTA= 30\degree$, then length of $TA$ is:
      \begin{enumerate}
          \item $2\sqrt3 cm$
          \item 2 cm
          \item $2\sqrt2$ cm
          \item $\sqrt3$ cm
      \end{enumerate}




  \begin{figure}[H]
	\centering
	\input{./figs/fig8.tex}
	\caption{}
	\label{fig.circle8}
  \end{figure}





\item In \figref{fig.circle9},$PT$ is a tangent at $\vec{T}$ to the circle with centre $\vec{O}$.If $\angle TPO=25\degree$, then $x$ is equal to:
   \begin{enumerate}
       \item 25\degree
       \item 65\degree
       \item 90\degree
       \item 115\degree
       
   \end{enumerate}




\begin{figure}[H]
	        \centering
	        \input{./figs/fig9.tex}
		\caption{}
		\label{fig.circle9}
\end{figure}






\item Two concentric circles are of radii 5 cm and 3 cm.Find the length of the cord of the larger circle which touches the smaller circle.

         
\end{enumerate}

