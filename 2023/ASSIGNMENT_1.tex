
\documentclass{article}
\usepackage{siunitx}
\usepackage{setspace}
\usepackage{gensymb}
\usepackage{xcolor}
\usepackage{caption}
%\usepackage{subcaption}
%\doublespacing
\singlespacing
%\usepackage[none]{hyphenat}
\usepackage{amssymb}
%\usepackage{relsize}
\usepackage[cmex10]{amsmath}
\usepackage{mathtools}
\usepackage{amsmath}
\usepackage{commath}
%\usepackage{amsthm}
%\interdisplaylinepenalty=2500
%\savesymbol{iint}
%\usepackage{txfonts}
%\restoresymbol{TXF}{iint}
%\usepackage{wasysym}
\usepackage{amsthm}
\usepackage{mathrsfs}
\usepackage{txfonts}
\let\vec\mathbf{}
%\usepackage{stfloats}
\usepackage{float}
\usepackage{cite}
\usepackage{cases}
\usepackage{subfig}
%\usepackage{xtab}
\usepackage{longtable}
\usepackage{multirow}
%\usepackage{algorithm}
\usepackage{amssymb}
%\usepackage{algpseudocode}
\usepackage{enumitem}
\usepackage{mathtools}
%\usepackage{eenrc}
%\usepackage[framemethod=tikz]{mdframed}
\usepackage{listings}
\usepackage{listings}
\usepackage[latin1]{inputenc}
%%\usepackage{color}{   
%%\usepackage{lscape}
\usepackage{textcomp}
\usepackage{titling}
\usepackage{hyperref}
%\usepackage{fulbigskip}   
\usepackage{tikz}
\usepackage{graphicx}
\usepackage{tfrupee}
\graphicspath{{figs/}}

\newcommand{\mydet}[1]{\ensuremath{\begin{vmatrix}#1\end{vmatrix}}}
\providecommand{\brak}[1]{\ensuremath{\left(#1\right)}}

\newcommand{\solution}{\noindent \textbf{Solution: }}
\newcommand{\myvec}[1]{\ensuremath{\begin{pmatrix}#1\end{pmatrix}}}
\let\vec\mathbf{}
\lstset{
  frame=single,
  breaklines=true
}




\begin{document}
\begin{center}
    \textbf{ \LaTeX{} Assignment Geometry}
\end{center}

\begin{enumerate}
    \item The hour-hand of a clock is $6$ cm long.The angle swept by it between $7:20$ a.m. and $7:55$ a.m. is:

\begin{enumerate}[label=(\alph*)]
    \item $\brak{\frac{35}{4}}\degree$
    \item $\brak{\frac{35}{2}}\degree$
    \item $35\degree$
    \item $70\degree$
\end{enumerate}

\item In the given \figref{fig:30_2_1_Q18}, $ AB \parallel PQ $.If $AB=6$ cm,$PQ=2$ cm and $OB=3$ cm,then the length of $OP$ is:
    
    \begin{figure}[!ht]
        \centering
        \includegraphics[width=\columnwidth]{figs/30_2_1_Q18.png}
        \caption{geometric figure}
        \label{fig:30_2_1_Q18}
    \end{figure}
    
\begin{enumerate}[label=(\alph*)]
    \item $9$cm
    \item $3$cm
    \item $4$cm 
    \item $1$cm
\end{enumerate}

\item The length of the shadow of a tower on the plane ground is $\sqrt{3}$ times the height of the tower.Find the angle of elevation of the sun.

\item  The angle of elevation of the top of a tower from a point on the ground which is $30$ m away from the foot of the tower,is $30\degree$ .Find the height of the tower.

\item  A car has two wipers which do not overlap. Each wiper has a blade of length $21$ cm sweeping through an angle of $120\degree$. Find the total area cleaned at each sweep of the two blades.

\item  As observed from the top of a $75$ m high lighthouse from the sea-level,the angles of depression of two ships are $30\degree$ and $60\degree$.If one ship is exactly behind the other on the same side of the lighthouse,find the distance between two ships.$\brak{Use \sqrt{3} = 1.73}$

\item  From a point on the ground,the angle of elevation of the bottom and top of a transmission tower fixed at the top of $30$ m high building are $30\degree$ and $60\degree$, respectively.Find the height of the transmission tower.$\brak{Use \sqrt{3} = 1.73}$
    
\item Sides $AB$ and $BC$ and median $AD$ of a triangle $ABC$ are respectively proportional to sides $PQ$ and $QR$ and median $PM$ of $\triangle PQR$. Show that $\triangle ABC \sim \triangle PQR$. 

\item  Through the mid-point $M$ of the side $CD$ of a parallelogram $ABCD$,the line $BM$ is drawn intersecting $AC$ in $L$ and $AD$(produced) in $E$.Prove that

    \begin{align}
    EL &= 2BL.
    \end{align}

\item  In an annual day function of a school,the organizers wanted to give a cash prize along with a memento to their best students. Each memento is made as shown in  the \figref{fig:30_2_1_Q36} and its base $ABCD$ is shown from the front side. The rate of silver plating is \rupee \hspace{4 pt}$20 \hspace{4 pt} per \hspace{4 pt}  cm^2$.

\begin{figure}[!ht]
    \centering
    \includegraphics[width=\columnwidth]{figs/30_2_1_Q36.png}
    \caption{memento}
    \label{fig:30_2_1_Q36}
\end{figure}

Based on the above, answer the following questions:
\begin{enumerate}[label=(\roman*)]
    \item What is the area of the quadrant $ODCO$?
    \item Find the area of $\triangle  AOB$.
    \item What is the total cost of silver plating the shaded part $ABCD$?
    \item what is the length of arc $CD$?
\end{enumerate}


\end{enumerate}
\end{document}











