%\documentclass[12pt,-letter paper]{article}
%\usepackage{siunitx}
%\usepackage{setspace}
%\usepackage{gensymb}
%\usepackage{xcolor}
%\usepackage{caption}
%\usepackage{subcaption}
%\doublespacing
%\singlespacing
%\usepackage[none]{hyphenat}
%\usepackage{amssymb}
%\usepackage{relsize}
%\usepackage[cmex10]{amsmath}
%\usepackage{mathtools}
%\usepackage{amsmath}
%\usepackage{commath}
%\usepackage{amsthm}
%\interdisplaylinepenalty=2500
%\savesymbol{iint}
%\usepackage{txfonts}
%\restoresymbol{TXF}{iint}
%\usepackage{wasysym}
%\usepackage{amsthm}
%\usepackage{mathrsfs}
%\usepackage{txfonts}
%\let\vec\mathbf{}
%\usepackage{stfloats}
%\usepackage{float}
%\usepackage{cite}
%\usepackage{cases}
%\usepackage{subfig}
%\usepackage{xtab}
%\usepackage{longtable}
%\usepackage{multirow}
%\usepackage{algorithm}
%\usepackage{amssymb}
%\usepackage{algpseudocode}
%\usepackage{enumitem}
%\usepackage{mathtools}
%\usepackage{eenrc}
%\usepackage[framemethod=tikz]{mdframed}
%\usepackage{listings}
%\usepackage{listings}
%\usepackage[latin1]{inputenc}
%\usepackage{color}{   
%\usepackage{lscape}
%\usepackage{textcomp}
%\usepackage{titling}
%\usepackage{hyperref}
%\usepackage{fulbigskip}   
%\usepackage{tikz}
%\usepackage{graphicx}
%\lstset{
  %frame=single,
  %breaklines=true
%}
%\let\vec\mathbf{}
%\usepackage{enumitem}
%\usepackage{graphicx}
%\usepackage{siunitx}
%\let\vec\mathbf{}
%\usepackage{enumitem}
%\usepackage{graphicx}
%\usepackage{enumitem}
%\usepackage{tfrupee}
%\usepackage{amsmath}
%\usepackage{amssymb}
%\usepackage{mwe} % for blindtext and example-image-a in example
%\usepackage{wrapfig}
%\graphicspath{{figs/}}
%\providecommand{\mydet}[1]{\ensuremath{\begin{vmatrix}#1\end{vmatrix}}}
%\providecommand{\myvec}[1]{\ensuremath{\begin{bmatrix}#1\end{bmatrix}}}
%\providecommand{\cbrak}[1]{\ensuremath{\left\{#1\right\}}}
%\providecommand{\sbrak}[1]{\ensuremath{{}\left[#1\right]}}
%\providecommand{\brak}[1]{\ensuremath{\left(#1\right)}}

%\begin{document}

%\title{Differentiation}

%\date{\today}

\begin{enumerate}
\item The order and degree of the differential equation of the family of parabolas having vertex at origin and axis along positive x-axis is
\begin{enumerate}
    \item $1,1$
    \item $1,2$
    \item $2,1$
    \item $2,2$
\end{enumerate}
\item If $y = \log x$, then $\frac{d^2y}{dx^2}$ =  \rule{30pt}{1pt}.
\item If $y = e^x + e^{-x}$, then show that $\frac{dy}{dx}$ = $\sqrt{y^2 - 4}$.

\item If $y=x^{\sin x }+\sin^{-1}(\sqrt x)$, the find $\frac{dy}{dx}$.
\item Find the intervals in which the function $f$ defined as $f(x) = \sin(x) + \cos(x)$, $0 \leq x \leq 2\pi$ is strictly increasing or decreasing.
\item Prove that the radius of the right circular cylinder of greatest curved surface area which can be inscribed in a given cone is half of that of the cone. 
\item $\lim\limits_{x \to 0}{\frac{e^{-x} - e^x}{x}}$ is equal to
\begin{enumerate}
    \item $2$
    \item $1$
    \item $-1$
    \item $2$
\end{enumerate}


\item A firm knows that the demand function for one of its product is linear. It also knows that it can sell $1400$ units when the price is \rupee{4} per unit and it can sell $1800$ units at a price \rupee{2} per unit.Find the marginal revenue function of this product.

\item Find the intervals in which the function $f(x) = x^{4}–4x^{3}+6x^{2}–4x+1$ is increasing or decreasing.

\item If $\sqrt{1-x^{2}}+\sqrt{1-y^{2}}=4(x-y)$,then show that $\frac{dy}{dx}=\frac{\sqrt{1-y^{2}}}{\sqrt{1-x^{2}}}$.


\item If $f'(x)=3x^{2}-4x-\frac{2}{x^{3}}$ and $f(1)=0$, then find $f(2)$.

\item A window is in the form of a rectangle mounted by a semi-circular opening. The total perimeter of the window is 10 m. Find the dimensions of the rectangular part of the window to admit maximum light through the whole opening.

\item Divide the number 8 into two positive numbers such that the sum of the cube of one and the square of the other is minimum.

\item If $e^{x}+e^{y}=e^{x+y}$, then $\frac{dy}{dx}$ is:
\begin{enumerate}
\item $e^{y-x}$
\item $e^{x+y}$
\item $-e^{y-x}$
\item $2 e^{x-y}$
\end{enumerate}

\item $If y=5\cos x-3\sin x$, then $\frac{d^{2}y}{dx^{2}}$ is equal to
\begin{enumerate}
\item -y
\item y
\item 25y
\item 9y
\end{enumerate}

\item The points on the curve $\frac{x^2}{9}+\frac{y^2}{16}=1$ at which the tangents are parallel to y-axis are:
\begin{enumerate}

\item $(0,\pm 4)$
\item $(\pm 4,0)$
\item $(\pm 3,0)$
\item $(0,\pm 3)$

\end{enumerate}

\end{enumerate}

%\end{document}
