\begin{enumerate}

        \item A quadrilateral ABCD is drawn to circumscribe a circle Fig.\ref{fig:leaf} \\ Prove that AB + CD = AD + BC.
    
        \begin{figure}[h]
        \centering
        \includegraphics[width=\columnwidth]{figs/2.jpg}
        \caption{A circle inscribed in a quadrilateral}
        \label{fig:leaf}
        \end{figure}
            
        \item Draw a pair of tangents to a circle of radius 4 cm which are inclined to each other at an angle of $40\degree$. 

        \item  A point T is 13 cm away from the centre of a circle. The length of the tangent drawn from T to the circle is 12 cm. Find the radius of the circle. 
    
        \item  Two tangents TP and TQ are drawn to a circle with centre O from an external point T. Prove that $\angle PTQ= 2 \angle OPQ$. 

        \item PQ is a tangent to a circle with centre O at the point P on the circle. If $\triangle OPQ$ is an isosceles triangle, then find $\angle OQB$. 

        \item Two concentric circles have radii 10 cm and 6 cm. Find the length of the chord of the larger circle which touches the smaller circle. 

        \item If tangents PA and PB from an external point P to a circle with centre O are inclined to each other at an angle of $70\degree$, then find $\angle POA$. 

        \item ABC is right triangle, right-angled at B, with BC = 6 cm and AB = 8 cm. A circle with centre O and radius r cm has been inscribed in $\triangle ABC$ as shown in the Fig.\ref{fig}. Find the value of r. 
  
        \begin{figure}[h]
        \centering
        \includegraphics[width=\columnwidth]{figs/3.jpg}
        \caption{A circle inscribed in a triangle}
        \label{fig}
        \end{figure}
        
        \item In a right triangle ABC, right-angled at B, BC = 6 cm and AB = 8 cm. A circle is inscribed in the $\triangle ABC$. Find the radius of the incircle. 
            
        \item Two circles touch externally at P and AB is a common tangent, touching one circle at A and the other at B. Find the measure of $\angle APB$.

        \item From an external point P, tangents PQ and PR are drawn to a circle with centre O, touching the circle at Q and R. If $\angle QOR$ = $140\degree$, find the measure of $\angle QPR$. 

        \item A circle touches all the sides of a quadrilateral ABCD. Prove that $AB + CD = DA + BC$.

        \item Write the steps of construction of a circle of diameter 6 cm and drawing of a pair of tangents to the circle from a point 5 cm away from the centre. 
            		
	\end{enumerate}	
