\begin{enumerate}
\item Check whether $13$ cm, $12$ cm, $5$ cm can be the sides of a right triangle.
\item 
\begin{enumerate}
    \item If a PL and PM are two tangents to a circle with center $\vec{O}$ from an external point $\vec{P}$ and $PL=4$ cm, find the length of OP, where radius of the circle is $3$ cm.
    \item Find the distance between two parallel tangents of a circle of radius $2\cdot5$ cm.
\end{enumerate}
    \item 
    \begin{enumerate}
    \item $\vec{D}$ and $\vec{E}$ are points on the sides $CA$ and $CB$ respectively of  a triangle $ABC$, right-angled at $\vec{C}$.
    
    Prove that $AE^2+BD^2=AB^2+DE^2$.
    
    \item Diagonals of a trapezium $ABCD$ with $AB\parallel DC$ intersect each other at the point $\vec{O}$. If $AB=2CD$, find the ratio of the areas of triangles $AOB$ and $COD$.
    \end{enumerate}

    \item Answer any \textbf{four} of the following questions :
      \begin{enumerate}[label=(\roman*)]
        \item Given $\triangle ABC \sim \triangle PQR$. If $\frac{AB}{PQ}=\frac{1}{3}$,then $\frac{ar(\triangle ABC)}{ar(\triangle PQR)}$ is 
        
        \begin{enumerate}[label=(\Alph*]
            \item $\frac{1}{3}$
            \item $3$
            \item $\frac{2}{3}$
            \item $\frac{1}{9}$
        \end{enumerate}
        
        \item The length of an altitude of an equilateral triangle of side $8$ cm is
        
          \begin{enumerate}[label=(\Alph*)]
            \item $4$ cm
            \item $4\sqrt{3}$ cm
            \item $\frac{8}{3}$ cm
            \item $12$ cm
        \end{enumerate}
        
        \item In $\triangle PQR$, $PQ=6\sqrt{3}$ cm, $PR=12 cm$ and $QR = 6$ cm. The measure of angle $\vec{Q}$ is
        
        \begin{enumerate}[label=(\Alph*)]
            \item 120\textdegree
            \item 60\textdegree
            \item 90\textdegree
            \item 40\textdegree
        \end{enumerate}
        
        \item If $\triangle ABC\sim\triangle PQR$ and $\angle B=46$\textdegree and $\angle R=69$\textdegree, then the measure of $\angle$A is
        
        \begin{enumerate}[label=(\Alph*)]
            \item 65\textdegree
            \item 111\textdegree
            \item 44\textdegree
            \item 115\textdegree
        \end{enumerate}
        
        \item $\vec{P}$ and $\vec{Q}$ are the points on the sides $AB$ and $AC$ respectively of a $\triangle ABC$ such that $PQ\parallel BC$. If $AP:PB=2:3$ and $AQ=4$ cm,then $AC$ is equal to

        \begin{enumerate}[label=(\Alph*)]
            \item $6$ cm
            \item $8$ cm
            \item $10$ cm
            \item $12$ cm
        \end{enumerate}
        \end{enumerate}
        \item Write the steps of construction of drawing a line segment $AB=4\cdot8$ cm and finding a point $\vec{P}$ on it such that $AP=\frac{1}{4}AB$.
        
        \item Answer any \textbf{four} of the following questions :
        \begin{enumerate}[label=(\roman*)]
        \item $ABC$ and $BDE$ are two equilateral triangles such that $\vec{D}$ is the mid-point of $BC$. The ratio of the areas of the triangles $ABC$ and $BDE$ is
        \begin{enumerate}[label=(\Alph*)]
            \item 2:1
            \item 1:2
            \item 4:1
            \item 1:4
        \end{enumerate}
        
        \item In $\triangle$ ABC , $AB=4\sqrt{3}$ cm, $AC=8$ cm and $BC=4$ cm. The angle $B$ is

        \begin{enumerate}[label=(\Alph*)]
            \item 120\textdegree
            \item 90\textdegree
            \item 60\textdegree
            \item 45\textdegree
        \end{enumerate}
         
        \item The perimeters of two similar triangles are $35$ cm and $21$ cm respectively.  If one side of the first triangle is $9$ cm, then the corresponding side of the second triangle is 
        
         \begin{enumerate}[label=(\Alph*)]
            \item $5\cdot4$ cm
            \item $4\cdot5$ cm
            \item $5\cdot6$ cm
            \item $15$ cm
        \end{enumerate}
        
        \item In a $\triangle ABC$ , $\vec{D}$ and $\vec{E}$ are points on the sides $AB$ and $AC$ respectively such that $DE\parallel BC$ and $AD:DB=3:1$. If $AE=3\cdot3 $ cm, then $AC$ is equal to
        
        \begin{enumerate}[label=(\Alph*)]
            \item $4$ cm
            \item $1\cdot1$ cm
            \item $4\cdot5$ cm
            \item $5\cdot5$ cm
        \end{enumerate}
        
        \item In an isosceles triangle $ABC$, if $AC=BC$ and $AB^2=2AC^2$, the $\angle$C is equal to
        \begin{enumerate}[label=(\Alph*)]
            \item 30\textdegree
            \item 45\textdegree
            \item 60\textdegree
            \item 90\textdegree
        \end{enumerate}
        \end{enumerate}
\end{enumerate}
