\iffalse

\let\negmedspace\undefined
\let\negthickspace\undefined
\documentclass{article}
%\documentclass[journal,12pt,twocolumn]{IEEEtran}
\usepackage{cite}
\usepackage{amsmath,amssymb,amsfonts,amsthm}
\usepackage{algorithmic}
\usepackage{graphicx}
\usepackage{textcomp}
\usepackage{tfrupee}
\usepackage{xcolor}
\usepackage{txfonts}
\usepackage{listings}
\usepackage{enumitem}
\usepackage{mathtools}
\usepackage{float}
\usepackage{gensymb}
\usepackage[breaklinks=true]{hyperref}
\usepackage{tkz-euclide} % loads  TikZ and tkz-base
\usepackage{listings}
\usepackage{gvv}
%
%\usepackage{setspace}
%\usepackage{gensymb}
%\doublespacing
%\singlespacing

%\usepackage{graphicx}
%\usepackage{amssymb}
%\usepackage{relsize}
%\usepackage[cmex10]{amsmath}
%\usepackage{amsthm}
%\interdisplaylinepenalty=2500
%\savesymbol{iint}
%\usepackage{txfonts}
%\restoresymbol{TXF}{iint}
%\usepackage{wasysym}
%\usepackage{amsthm}
%\usepackage{iithtlc}
%\usepackage{mathrsfs}
%\usepackage{txfonts}
%\usepackage{stfloats}
%\usepackage{bm}
%\usepackage{cite}
%\usepackage{cases}
%\usepackage{subfig}
%\usepackage{xtab}
%\usepackage{longtable}
%\usepackage{multirow}
%\usepackage{algorithm}
%\usepackage{algpseudocode}
%\usepackage{enumitem}
%\usepackage{mathtools}
%\usepackage{tikz}
%\usepackage{circuitikz}
%\usepackage{verbatim}
%\usepackage{tfrupee}
%\usepackage{stmaryrd}
%\usetkzobj{all}
%    \usepackage{color}                                            %%
%    \usepackage{array}                                            %%
%    \usepackage{longtable}                                        %%
%    \usepackage{calc}                                             %%
%    \usepackage{multirow}                                         %%
%    \usepackage{hhline}                                           %%
%    \usepackage{ifthen}                                           %%
  %optionally (for landscape tables embedded in another document): %%
%    \usepackage{lscape}     
%\usepackage{multicol}
%\usepackage{chngcntr}
%\usepackage{enumerate}

%\usepackage{wasysym}
%\documentclass[conference]{IEEEtran}
%\IEEEoverridecommandlockouts
% The preceding line is only needed to identify funding in the first footnote. If that is unneeded, please comment it out.

\newtheorem{theorem}{Theorem}[section]
\newtheorem{problem}{Problem}
\newtheorem{proposition}{Proposition}[section]
\newtheorem{lemma}{Lemma}[section]

\newtheorem{corollary}[theorem]{Corollary}
\newtheorem{example}{Example}[section]
%\newtheorem{definition}[problem]{Definition}
%\newtheorem{thm}{Theorem}[section] 
%\newtheorem{defn}[thm]{Definition}
%\newtheorem{algorithm}{Algorithm}[section]
%\newtheorem{cor}{Corollary}
\newcommand{\BEQA}{\begin{eqnarray}}
\newcommand{\EEQA}{\end{eqnarray}}
%\newcommand{\define}{\stackrel{\triangle}{=}}
\theoremstyle{remark}
\newtheorem{rem}{Remark}

%\bibliographystyle{ieeetr}
\begin{document}
%

%\bibliographystyle{IEEEtran}


%\vspace{3cm}

%\title{
%	\logo{
%Sample \LaTeX Document with a Figure
%	}
%}
%\author{ G V V Sharma$^{*}$% <-this % stops a space
%	\thanks{*The author is with the Department
%%		of Electrical Engineering, Indian Institute of Technology, Hyderabad
%%		502285 India e-mail:  gadepall@iith.ac.in. All content in this manual is released under GNU GPL.  Free and open source.}
	
%}	
%\title{
%	\logo{Matrix Analysis through Octave}{\begin{center}\includegraphics[scale=.24]{tlc}\end{center}}{}{HAMDSP}
%}


% paper title
% can use linebreaks \\ within to get better formatting as desired
%\title{Matrix Analysis through Octave}
%
%
% author names and IEEE memberships
% note positions of commas and nonbreaking spaces ( ~ ) LaTeX will not break
% a structure at a ~ so this keeps an author's name from being broken across
% two lines.
% use \thanks{} to gain access to the first footnote area
% a separate \thanks must be used for each paragraph as LaTeX2e's \thanks
% was not built to handle multiple paragraphs
%

%\author{<-this % stops a space
%\thanks{}}
%}
% note the % following the last \IEEEmembership and also \thanks - 
% these prevent an unwanted space from occurring between the last author name
% and the end of the author line. i.e., if you had this:
% 
% \author{....lastname \thanks{...} \thanks{...} }
%                     ^------------^------------^----Do not want these spaces!
%
% a space would be appended to the last name and could cause every name on that
% line to be shifted left slightly. This is one of those "LaTeX things". For
% instance, "\textbf{A} \textbf{B}" will typeset as "A B" not "AB". To get
% "AB" then you have to do: "\textbf{A}\textbf{B}"
% \thanks is no different in this regard, so shield the last } of each \thanks
% that ends a line with a % and do not let a space in before the next \thanks.
% Spaces after \IEEEmembership other than the last one are OK (and needed) as
% you are supposed to have spaces between the names. For what it is worth,
% this is a minor point as most people would not even notice if the said evil
% space somehow managed to creep in.



% The paper headers
%\markboth{Journal of \LaTeX\ Class Files,~Vol.~6, No.~1, January~2007}%
%{Shell \MakeLowercase{\textit{et al.}}: Bare Demo of IEEEtran.cls for Journals}
% The only time the second header will appear is for the odd numbered pages
% after the title page when using the twoside option.
% 
% *** Note that you probably will NOT want to include the author's ***
% *** name in the headers of peer review papers.                   ***
% You can use \ifCLASSOPTIONpeerreview for conditional compilation here if
% you desire.




% If you want to put a publisher's ID mark on the page you can do it like
% this:
%\IEEEpubid{0000--0000/00\$00.00~\copyright~2007 IEEE}
% Remember, if you use this you must call \IEEEpubidadjcol in the second
% column for its text to clear the IEEEpubid mark.



% make the title area
%\maketitle

%\newpage

%\tableofcontents

%\bigskip

\renewcommand{\thefigure}{\theenumi}
\renewcommand{\thetable}{\theenumi}
%\renewcommand{\theequation}{\theenumi}

%\begin{abstract}
%%\boldmath
%In this letter, an algorithm for evaluating the exact analytical bit error rate  (BER)  for the piecewise linear (PL) combiner for  multiple relays is presented. Previous results were available only for upto three relays. The algorithm is unique in the sense that  the actual mathematical expressions, that are prohibitively large, need not be explicitly obtained. The diversity gain due to multiple relays is shown through plots of the analytical BER, well supported by simulations. 
%
%\end{abstract}
% IEEEtran.cls defaults to using nonbold math in the Abstract.
% This preserves the distinction between vectors and scalars. However,
% if the journal you are submitting to favors bold math in the abstract,
% then you can use LaTeX's standard command \boldmath at the very start
% of the abstract to achieve this. Many IEEE journals frown on math
% in the abstract anyway.

% Note that keywords are not normally used for peerreview papers.
%\begin{IEEEkeywords}
%Cooperative diversity, decode and forward, piecewise linear
%\end{IEEEkeywords}



% For peer review papers, you can put extra information on the cover
% page as needed:
% \ifCLASSOPTIONpeerreview
% \begin{center} \bfseries EDICS Category: 3-BBND \end{center}
% \fi
%
% For peerreview papers, this IEEEtran command inserts a page break and
% creates the second title. It will be ignored for other modes.
%\IEEEpeerreviewmaketitle

%\begin{abstract}
%This manual includes \LaTeX figures.
%book provides an introduction to optimization  based on the NCERT textbooks from Class 6-12.  Links to sample Python codes are available in the text.  
%\end{abstract}
%Download 
%\begin{lstlisting}
%svn co https://github.com/gadepall/school/trunk/training
%\end{lstlisting}

%\renewcommand{\theequation}{\theenumi}
%\subsection{Problem}

%\section{Chemistry}
%\begin{enumerate}[label=\arabic*.,ref=\thesection.\theenumi]
%\numberwithin{equation}{enumi}
%\begin{enumerate}[label=\arabic*.,ref=\theenumi]
%\numberwithin{equation}{enumi}
%\item Fig. \ref{fig:tri_sss_py} is generated using 
%\begin{lstlisting}
%math/codes/tri_sss.py
%\end{lstlisting}
%
%\begin{figure}
%\centering
%\includegraphics[width=\columnwidth]{./figs/tri_sss.pdf}
%\caption{Triangle generated using python}
%\label{fig:tri_sss_py}
%\end{figure}
%
%\item Fig. \ref{fig:tri_sss_tikz} is generated using 
%\begin{lstlisting}
%math/figs/tri_sss_alone.tex
%\end{lstlisting}
%\begin{figure}[!ht]
%	\begin{center}
%		
%		\resizebox{\columnwidth}{!}{\input{./figs/tri_sss.tex}}
%	\end{center}
%	\caption{Triangle generated using \LaTeX Tikz.}
%	\label{fig:tri_sss_tikz}	%
%\end{figure}

%\end{enumerate}
%\end{document}
%\maketitle {SECTION A}
\fi

\begin{enumerate}
    \item If $\sin \theta=0$, then the value of $\tan^2\theta+\cot^2\theta$ is
    \begin{enumerate}
        \item $2$
        \item $4$
        \item $1$
        \item $\frac{10}{9}$
    \end{enumerate}
    \item The value(s) of $k$ for which the quadratic equation 
    \begin{align}
        3x^2 - kx + 3 = 0
    \end{align}
    has equal roots, is (are) 
    \begin{enumerate}
        \item $6$
        \item $-6$
        \item $\pm6$
        \item $9$
    \end{enumerate}
    \item $5\tan^2 \theta - 5\sec^2\theta = \underline{\hspace{2cm}}$
    \item If $\alpha$, $\beta$ are zeroes of the polynomial $2x^2 - 5x - 4$, then $\frac{1}{\alpha}+\frac{1}{\beta}$.
    \item In  \figref{fig:as.jpeg}, a tower stands vertically on the ground. From a point on the ground, which is $80m$ away from the foot of the tower, the angle of elevation of the tower is found to be $30\degree$. Find the height of the tower.
    \begin{figure}[H]
        \centering
        \includegraphics[width=70mm]{figs/as.jpeg}
        \caption{as.jpeg}
        \label{fig:as.jpeg}
    \end{figure}
    \item Solve
    \begin{align}
        9x^2 - 6a^2x + a^4 - b^4 = 0
    \end{align}
    using the quadratic formula.
    \item Show that 
    \begin{align}
        \cos(38\degree) \cos(52\degree) - \sin(38\degree)\sin(52\degree) = \cos(90\degree).
    \end{align}
    \item Prove that 
    \begin{align}
        \frac{\sin\theta}{\cot\theta+\csc\theta} = 2+\frac{\sin\theta}{\cot\theta-\csc\theta}.
    \end{align}
    \item Given 
    \begin{align}
        15 \cot (A) = 8,
    \end{align}
    find the values of $\sin (A)$ and $\sec (A)$.
    \item The angles of depression of the top and bottom of a tower as seen from the top of a $60\sqrt{3}m$ high cliff are $45\degree$ and $60\degree$ respectively. Find the height of the tower. (Use $\sqrt{3}=1.73$)
    \item $A$ and $B$ jointly finish a piece of work in $15$ days. When they work separately, $A$ takes $16$ days less than the number of days taken by $B$ to finish the same piece of work. Find the number of days taken by $B$ to finish the work.
    \item If the polynomial
    \begin{align} 
        f(x) = 3x^4 - 9x^3 + x^2 + 15x + k
    \end{align}
    is completely divisible by $3x^2 - 5$, then find the value of $k$. Using the quotient obtained, find two zeroes of the polynomial.
    \item Find all the zeroes of the polynomial
    \begin{align}
        f(x)x^4 - 8x^3 + 23x^2 - 28x + 12
    \end{align}
    if two of its zeroes are $2$ and $3$.  
    \item Find the value of $m$ for which the quadratic equation
    \begin{align}
        (m-1)x^2 + 2(m-1)x + 1 = 0
    \end{align}
    has two real and equal roots. 
    \item Solve the following quadratic equation for $x$ 
    \begin{align}
        \sqrt{3}x^2 + 10x + 7\sqrt{3} = 0
    \end{align}
    \item  The product of Rehan's age (in years) $5$ years ago and his age $7$ years from now is one more than twice his age. Find his present age.
    \item The angle of elevation of the top of a building from the foot of the tower is $30\degree$ and the angle of elevation of the top of the tower from the foot of the building is $60\degree$. If the tower is $50$ meters high, then find the height of the building.
    \item From a point on a bridge across a river, the angles of depression of the banks on opposite sides of the river are $30\degree$ and $60\degree$ respectively. If the bridge is at a height of $3$ meters from the banks, then find the width of the river. 
    \item In \figref{fig:ak}, Gadisar Lake is located in the Jaisalmer district of Rajasthan. It was built by the King of Jaisalmer and rebuilt by Gadsi Singh in the $14$th century. The lake has many Chhatris. One of them is shown below:
    \begin{figure}[H]
        \centering
    	 \includegraphics[width=70mm]{figs/ak.jpeg}
        \caption{ak.jpg}
        \label{fig:ak}
    \end{figure}
    Observe the picture. From a point $A$ $h$ meters above the water level, the angle of elevation of the top of Chhatri (point $B$) is $45\degree$ and the angle of depression of its reflection in the water (point $C$) is $60\degree$ . If the height of Chhatri above water level is (approximately) $10$ meters, then 
    \begin{enumerate}
        \item Draw a well-labeled figure based on the above information.
        \item Find the height ($h$) of the point $A$ above water level. (Use $\sqrt{3}=1.73$) 
    \end{enumerate}

    \item Solve the quadratic equation 
    \begin{align}
        x^2 + \sqrt{2}x - 6 = 0
    \end{align}
    for $x$.
    
    \item In \figref{fig:su.jpeg}, from a point on a bridge across a river, the angles of depression of the banks on opposite sides of the river are $30\degree$ and $45\degree$. If the bridge is at a height of $8$ meters from the banks, then find the width of the river.
    \begin{figure}[H]
        \centering
        \includegraphics[width=70mm]{figs/su.jpeg}
        \caption{su.jpg}
        \label{fig:su.jpeg}
    \end{figure}
    
    \item A $2$-digit number is such that the product of its digits is $24$. If $18$ is subtracted from the number, the digits interchange their places. Find the numbers.
    
    \item The difference of the squares of two numbers is $180$. The square of the smaller number is $8$ times the greater number. Find the two numbers.
    
    \item Case Study-1:
    
    In \figref{fig:kite.jpeg}, Kite Festival is celebrated in many countries at different times of the year. In India, every year on $14^{th}$ January is celebrated as International Kite Day. On this day, many people visit India and participate in the festival by flying various kinds of kites.
    
    \begin{figure}[H]
	\centering
        \includegraphics[width=70mm]{figs/kite.jpeg}
        \caption{kites}
        \label{fig:kite.jpeg}
    \end{figure}
    
    In Fig. 5, the angles of elevation of two kites (Point $A$ and $B$) from the hands of a man (Point $C$) are found to be $30\degree$ and $60\degree$ respectively. Taking $AD = 50$ meters and $BE = 60$ meters, find:
    \begin{enumerate}
        \item The lengths of strings used (take them straight) for kites $A$ and $B$ as shown in the figure.
        \item The distance $d$ between these two kites.
    \end{enumerate}
    
    \item Solve the quadratic equation for $x$:
    \begin{align}
        x^2 - 2ax - (4b^2 - a^2) = 0
    \end{align}
    
    \item If the quadratic equation
    \begin{align}
        (1+a^2)x^2 + 2abx + (b^2-c^2) = 0
    \end{align}
    has equal and real roots, then prove that:
    \begin{align}
        b^2 = c^2(1+a^2)
    \end{align}
    
    \item Two boats are sailing in the sea $80$ meters apart from each other towards a cliff $AB$. The angles of depression of the boats from the top of the cliff are $30\degree$ and $45\degree$ respectively, as shown in \figref{fig:boat.jpeg}
    
    \begin{figure}[H]
        \centering
        \includegraphics[width=70mm]{figs/boat.edit.jpeg}
        \caption{boat}
        \label{fig:boat.jpeg}
    \end{figure}
    
    Find the height of the cliff.
    
    \item The angle of elevation of the top $Q$ of a vertical tower $PQ$ from a point $X$ on the ground is $60\degree$. From a point $Y$, $40$ meters vertically above $X$, the angle of elevation of the top $Q$ of tower $PQ$ is $45\degree$. Find the height of the tower $PQ$ and the distance $PX$. (Use $\sqrt{3} = 1.73$)
    
    \item Find the value of $k$ for which the quadratic equation
    \begin{align}
        2kx^2 - 40 + 25 = 0
    \end{align}
    has real and equal roots. 
    
    \item Solve for $x$:
    \begin{align}
        \frac{5}{2}x^2 + \frac{2}{5} = 1 - 2x
    \end{align}
    
    \item An Aeroplane at an altitude of $200$ meters observes the angles of depression of opposite points on the two banks of a river to be $45\degree$ and $60\degree$. Find the width of the river. (Use $\sqrt{3} = 1.732$)
    
    \item Find the value(s) of $'p'$ for which the quadratic equation $(px-4)(x-2)$ has real and equal roots.
    
    \item Had Aarush scored $8$ more marks in a Mathematics test, out of $35$ marks, $7$ times these marks would have been $4$ less than the square of his actual marks. How many marks did he get in the test?
    
    \item From the top of an $8$ meter high building, the angle of elevation of the top of a cable tower is $60\degree$ and the angle of depression of its foot is $45\degree$. Determine the height of the tower. (Take $\sqrt{3} = 1.732$).
    
    \item Find the roots of the quadratic equation 
    \begin{align}
        9x^2 - 6\sqrt{2}x + 2 = 0
    \end{align}
    
    \item The product of two consecutive odd positive integers is $255$. Find the integers, by formulating a quadratic equation.
    
    \item Find the value(s) of $k$ for the quadratic equation,
    \begin{align}
        (k+3)x^2 + kx + 1 = 0
    \end{align}
    to have two real and equal roots.
    
    \item As observed from the top of a lighthouse $60$ meters high from the sea level, the angles of depression of two ships are $45\degree$ and $60\degree$. If one ship is exactly behind the other on the same side of the lighthouse, then find the distance between the two ships. (Use $\sqrt{3} = 1.732$)
    
    \item At a point on the level ground, the angle of elevation of the top of a vertical tower is found to be $\alpha$, such that $\tan\alpha = \frac{5}{12}$. On walking $192$ meters towards the tower, the angle of elevation $\beta$ is such that $\tan\beta = \frac{3}{4}$. Find the height of the tower.
    
    \item $\tan^{-1}\frac{1}{\sqrt{3}} - \cot^{-1}\frac{-1}{\sqrt{3}}$
    
    \item Show that the relation $R$ in the set of all real numbers, defined as $R = \{(a,b) : a \leq b^2\}$.
    
    \item Two angles of a triangle are $\cot^{-1}2$ and $\cot^{-1}3$. The third angle of the triangle is?
    
    \item Solve for $x$:
    \begin{align}
        \sin^{-1}(1-x) - 2 \sin^{-1} x = \frac{\pi}{2}
    \end{align}
    \item Find the present value of a perpetuity of \rupee~18,000 at the end of $6$ months if it is worth $8\%$ p.a. compounded semi-annually.
    
    [Given that: $1.00833^{12} = 1.1047$]
    
    \item Find the effective rate which is equivalent to a nominal rate of $10\%$ p.a. compounded monthly.
    
    [Given that: $1.00833^{12} = 1.1047$]
    
    \item Abhay bought a mobile phone for \rupee~30,000. The mobile phone is estimated to have a scrap value of \rupee~3,000 after a span of $3$ years. Using the linear depreciation method, find the book value of the mobile phone at the end of $2$ years.
    
    \item Madhu exchanged her old car valued at \rupee~1,50,000 with a new one priced at \rupee~65,000. She paid \rupee~x as a down payment, and the balance in $20$ monthly equal installments of \rupee~21,000. The rate of interest offered to her is $9\%$ p.a. Find the value of $x$.
    
    [Given that: $1.0075^{-20} = 0.86118985$]
    
    \item Calculate the EMI under the 'Flat Rate System' for a loan of $\rupee~5,00,000$ with $10\%$ annual interest rate for $5$ years.
    
    \item A machine costing \rupee~2,00,000 has an effective life of $7$ years, and its scrap value is \rupee~30,000. What amount should the company put into a sinking fund earning $5\%$ p.a., so that it can replace the machine after its usual life? Assume that a new machine will cost \rupee~3,00,000 after $7$ years.
    
    [Given that: $(1.05)^7 = 1.407$]
    
    \item A start-up company invested \rupee~3,00,000 in shares for $5$ years. The value of this investment was \rupee~3,50,000 at the end of the second year, \rupee~3,80,000 at the end of the third year, and on maturity, the final value stood at \rupee~4,50,000.Calculate the Compound Annual Growth Rate (CAGR) on the investment.
    
    [Given that: $(1.5)^\frac{1}{5} = 1.084$]
\end{enumerate}
\end{document}
