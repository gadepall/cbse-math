\documentclass[12pt]{article}
\usepackage{hyperref}
\usepackage{listings}
\usepackage{biblatex}
\usepackage{tikz}
\usepackage{refstyle}
\usepackage{mathabx}
\usepackage{amssymb}
\usepackage{amsmath}
\usepackage{caption}
\usepackage{float}
\usepackage{graphicx}
\usepackage{graphics}
\usepackage{subfig}
\usepackage{tfrupee}
\newcommand{\abs}[1]{\lvert#1\rvert}

\usepackage{siunitx}
\usepackage{setspace}
\usepackage{gensymb}
\usepackage{xcolor}
\usepackage{caption}
%\usepackage{subcaption}
\doublespacing
\singlespacing
\usepackage[none]{hyphenat}
\usepackage{amssymb}
\usepackage{relsize}
\usepackage[cmex10]{amsmath}
%\usepackage{mathtools}
\usepackage{amsmath}
\usepackage{commath}
\usepackage{amsthm}
\interdisplaylinepenalty=2500
%\savesymbol{iint}
\usepackage{txfonts}
%\restoresymbol{TXF}{iint}
\usepackage{wasysym}
\usepackage{amsthm}
\usepackage{mathrsfs}
\usepackage{txfonts}
\let\vec\mathbf{}
\usepackage{stfloats}
\usepackage{float}
\usepackage{cite}
\usepackage{cases}
\usepackage{subfig}
%\usepackage{xtab}
\usepackage{longtable}
\usepackage{multirow}
%\usepackage{algorithm}
\usepackage{amssymb}
%\usepackage{algpseudocode}
\usepackage{enumitem}
\usepackage{mathtools}
%\usepackage{eenrc}
%\usepackage[framemethod=tikz]{mdframed}
\usepackage{listings}
%\usepackage{listings}
\usepackage[latin1]{inputenc}
%%\usepackage{color}{   
%%\usepackage{lscape}
\usepackage{textcomp}
\usepackage{titling}
\usepackage{hyperref}
%\usepackage{fulbigskip}   
\usepackage{tikz}
\usepackage{graphicx}
\lstset{
  frame=single,
  breaklines=true
}
\let\vec\mathbf{}
\usepackage{enumitem}
\usepackage{graphicx}
\usepackage{siunitx}
\let\vec\mathbf{}
\usepackage{enumitem}
\usepackage{graphicx}
\usepackage{enumitem}
\usepackage{tfrupee}
\usepackage{amsmath}
\usepackage{amssymb}
\usepackage{mwe} % for blindtext and example-image-a in example
\usepackage{wrapfig}
\graphicspath{{figs/}}
\providecommand{\mydet}[1]{\ensuremath{\begin{vmatrix}#1\end{vmatrix}}}
\providecommand{\myvec}[1]{\ensuremath{\begin{bmatrix}#1\end{bmatrix}}}
\providecommand{\cbrak}[1]{\ensuremath{\left\{#1\right\}}}
\providecommand{\brak}[1]{\ensuremath{\left(#1\right)}}
\graphicspath{{/storage/self/primary/Download/latexnew/fig}}
\title{Optimization assignment}
%\author{Prof.G V V sharma}
\date{\today}

\begin{document}

\maketitle{Questions}

\begin{enumerate}

\item If the corner points $(3,4)$ and $(5,0)$ of the feasible region in an LPP, give the same maximum value for the objective function $z=ax+by$, where a,b $>$ 0, then we have 


\begin{enumerate}
\item $a=2b$
\item $2a=b$
\item $2a=3b$
\item $3b=2a$
\end{enumerate}

\item A dietician wishes to mix two types of foods in such a way that vitamin contents of the mixture contain at least 8 units of vitamin A and 10 units of vitamin C. Food I contains 2 units/kg of vitamin A and 1 unit/kg of vitamin C. Food II contains 1  unit/kg of vitamin A and 2 units/kg of vitamin C. It costs 1 \rupee 50 per kg to purchase Food I and  \rupee 70 per kg to purchase Food II. Formulate this problem as a Linear Programming Problem for minimizing the cost of such a mixture.

\item Show that of all the rectangles inscribed in a given fixed circle, the square has maximum area.



Find the intervals in which the function f given by $f(x) = \sin x + \cos x, 0 \le x \le 2\pi$ is strictly increasing or strictly decreasing.

\item A company produces two types of goods, A and B that require gold and silver. Each unit of type A requires 3 g of silver and 1 g of gold, while that of type B requires 1 g of silver and 2 g of gold. The company can use at the most 9 g of silver and 8 g of gold. If each unit of type A brings a profit of    \rupee  120  and that of type B   \rupee 150  , then find the number of units of each type that the company should produce to maximize profit.

Formulate the above LPP and solve it graphically. Also, find the maximum profit.


\item Find the intervals in which the function f defined as $f(x) = \sin x + \cos x, 0 \le x \le 2\pi$ is strictly increasing or decreasing.



Prove that the radius of the right circular cylinder of greatest curved surface area which can be inscribed in a given cone is half of that of the cone.

\item Maximize $z=3x + 4y$, if possible,

subject to the constraints :
\begin{align}
 x - y &\le - 1 \\
-x + y &\le 0 \\
x, y &\ge 0
\end{align}


\item A dietician wishes to mix two types of foods F1 and F2 in such a way that the vitamin content of the mixture contains at least 8 units of vitamin A and 10 units of vitamin C. Food F1 contains 2 units/kg of vitamin A  and 1 unit/kg of vitamin C, while Food F2 contains 1 unit/kg of vitamin A and 2units/kg of vitamin C. It costs \rupee  5  per kg to purchase Food F1 and  \rupee  7  per kg to purchase Food F2


Based on the above information, answer the following questions:


\begin{enumerate}
\item To find out the minimum cost of such a mixture, formulate the above problem as a LPP.
\item Determine the minimum cost of the mixture.
\end{enumerate}


\item Find the area bounded by the curves $y = \abs{x - 1}$ and $y = 1$, using integration.

\end{enumerate}
\end{document}
