\begin{enumerate}

	\item Draw a circle of radius 2.5 cm. Take a point $\vec{P}$ outside the circle at a distance of 7 cm from the center. Then construct a pair of tangents to the circle from point $\vec{P}$.

	\item Write the steps of construction for constructing a pair of tangents to a circle of radius 4 cm from a point $\vec{P}$, at a distance of 7 cm from its center $\vec{O}$.

	\item In Figure \ref{fig:tan1}, there are two concentric circles with centre $\vec{O}$. If $ARC$ and $AQB$ are tangents to the smaller circle  from the point $\vec{A}$ lying on the larger circle, find the length of $AC$, if $AQ$ = 5 cm.
		\begin{figure}[H]
			\centering
			\includegraphics[width=\columnwidth]{figs/tan}
			\caption{Two concentric circles with $\vec{O}$ as centre}
			\label{fig:tan1}
		\end{figure}
	
	\item In Figure \ref{fig:cir1}, if a circle touches the side $QR$ of $\Delta PQR$ at $\vec{S}$ and extended sides $PQ$ and $PR$ at $\vec{M}$ and $\vec{N}$, respectively,
		\begin{figure}[H]
			\centering
			\includegraphics[width=\columnwidth]{figs/cir}
				\caption{Two tangents are drawn from point $\vec{P}$ to the circle}
				\label{fig:cir1}
		\end{figure}
		prove that $PM=\dfrac{1}{2}(PQ+QR+PR)$

	\item In Figure \ref{fig:tri1}, a triangle $ABC$ is drawn to circumscribe a circle of radius 4 cm such that the segments $BD$ and $DC$ into which $BC$ is divided by the point of contact $\vec{D}$ are of lengths 6 cm and 8 cm respectively. If the area of $\Delta ABC$ is 84 $cm^2$, find the lengths of sides $AB$ and $AC$.
		\begin{figure}[H]
			\centering
			\includegraphics[width=\columnwidth]{figs/tri}
				\caption{Circle with $\vec{O}$ as center circumscribed in triangle $ABC$}
				\label{fig:tri1}
		\end{figure}

	\item In Figure \ref{fig:sq1}, $PQ$ and $PR$ are tangents to the circle centered at $\vec{O}$. If $\angle OPR=45\degree$, then prove that $ORPQ$ is a square.
		\begin{figure}[H]
			\centering
			\includegraphics[width=\columnwidth]{figs/sq}
			\caption{Two tangents drawn from point $\vec{P}$ to a circle whose centre is $\vec{O}$}
			\label{fig:sq1}
		\end{figure}

	\item In Figure \ref{fig:sct1}, $\vec{O}$ is the centre of a circle of radius 5 cm. $PA$ and $BC$ are tangents to the circle at $\vec{A}$ and $\vec{B}$ respectively. If $OP$ is 13 cm, then find the length of tangents $PA$ and $BC$.
		\begin{figure}[H]
			\centering
			\includegraphics[width=\columnwidth]{figs/sct}
			\caption{Two tangents drawn from point $\vec{C}$ to a circle whose centre is $\vec{O}$}
			\label{fig:sct1}
		\end{figure}

	\item In Figure \ref{fig:ver1}, $AB$ is diameter of a circle centered at $\vec{O}$. $BC$ is tangent to the circle at $\vec{B}$. If $OP$ bisects the chord $AD$ and $\angle AOP=60\degree$, then find $m\angle C$.
		\begin{figure}[H]
			\centering
			\includegraphics[width=\columnwidth]{figs/ver}
			\caption{Tangent $BC$ is drawn from point $\vec{C}$ to a circle whose centre is $\vec{O}$}
			\label{fig:ver1}
		\end{figure}

	\item In Figure \ref{fig:hor1}, $XAY$ is a tangent to the circle centered at $\vec{O}$. If $\angle ABO=60\degree$,then find $m\angle BAY$ and $m\angle AOB$.
		\begin{figure}
			\centering
			\includegraphics[width=\columnwidth]{figs/hor}
			\caption{The line $XAY$ is tangent to the circle centered at $\vec{O}$}
			\label{fig:hor1}
		\end{figure}

	\item Two concentric circles are of radii 4cm and 3 cm. Find the length of the chord of the larger circle which touches the smaller circle.

	\item In Figure \ref{fig:sl1}, a triangle $ABC$ with $\angle B=90\degree$ is shown. Taking $AB$ as diameter, a circle has been drawn intersecting $AC$ at point $\vec{P}$. Prove that the tangent drawn at point $\vec{P}$ bisects $BC$.
		\begin{figure}[H]
			\centering
			\includegraphics[width=\columnwidth]{figs/sl}
			\caption{$PQ$ is tangent to the circle centered at $\vec{O}$. $AB$ is the diameter and $\angle B=90\degree$}
			\label{fig:sl1}
		\end{figure}
\item Find the equation of tangent to the curve $y = x^2 + 4x + 1$ at the point $(3,22)$.
\end{enumerate}
